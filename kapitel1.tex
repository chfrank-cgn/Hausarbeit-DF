%
%	Einfuehrung
%

\pagebreak
\section{Introduction}

\onehalfspacing

\subsection{Cyber Defense}

Cybersecurity, also known as information security or IT security, is the practice of protecting systems, networks, and programs from digital attacks. Its goal is to reduce the risk of these attacks and prevent the unauthorized exploitation of systems, networks, and technology.

Cybersecurity is an ongoing process because threats and attack vectors are constantly evolving. Organizations and individuals must proactively implement security measures and stay up-to-date on the latest threats.\footnote{See \textit{Gemini (2024)}: What is Cyber Security. \cite{bardCybersec}}

\subsection{Micro-Service Applications}

NIS2, which stands for Network and Information Systems Directive II, is the EU's legislative act to strengthen cybersecurity across the European Union. It's essentially an update to the original NIS Directive implemented in 2016.

It sets stricter requirements for various sectors to improve security for essential entities. These entities include organizations in critical sectors like energy, transport, waste management, healthcare, and digital infrastructure providers.\footnote{See \textit{NIS 2 Compliant.org (2024)}: Comprehensive Guide to the NIS 2 Directive. \cite{nisGuide}}

Compared to the original directive, NIS2 applies to a broader range of businesses and organizations; it recognizes the importance of securing the supply chain and includes measures to address vulnerabilities in third-party vendors and suppliers.

It also emphasizes a risk-based approach to cybersecurity. Organizations must identify and assess their security risks and implement appropriate mitigation measures.


\subsection{TLS and mTLS}



\subsection{Kubernetes}

Kubernetes, or K8s, is an open-source system designed to automate deploying, scaling, and managing applications built using containers. Containers package software in a standardized unit that includes all the dependencies it needs to run, such as code, libraries, and settings. This makes them portable and efficient.

Kubernetes helps manage these containers by grouping them logically. This makes it easier to track and manage complex applications with many containers. The original inspiration for Kubernetes came from Google's internal container orchestration system, Borg.\footnote{See \textit{Gemini (2024)}: What is Kubernetes. \cite{bardKubernetes}} 

Kubernetes reached the 1.0 milestone in 2015 and was donated to the CNCF in 2016. Its current release is 1.32, but 1.30 was a very special release:

"For the people who built it, for the people who release it, and for the furries who keep all of our clusters online, we present to you Kubernetes v1.30: Uwubernetes, the cutest release to date."\footnote{\textit{Dsouza, A. (2024)}: Kubernetes 1.30. \cite{uwubernetes}}

\begin{figure}[H]
\centering
\caption {Kubernetes 1.30 Release Logo}
\includegraphics[width=0.3\linewidth]{images/k8s-1.30.png}
\label{fig:uwubernetes}
\end{figure}

\subsection{Research Question \& Method}

This paper will examine.

To do this, we will perform an Experiment and cross-check the findings of two Service-Mesh installations on a Kubernetes cluster.\footnote{See \textit{Genau, L. (2020)}: Ein Experiment in Deiner Abschlussarbeit Durchführen. \cite{expScribbr}}

The goal is to establish whether LinkerD or Istio can add mTLS to an existing application and determine which uses fewer resources.

\subsection{Gender-neutral Pronouns}

Our society is becoming more open, inclusive, and gender-fluid, and now I think it's time to think about using gender-neutral pronouns in scientific texts, too. Two well-known researchers, Abigail C. Saguy and Juliet A. Williams, both from UCLA, propose to use the singular they/them instead: "The universal singular they is inclusive of people who identify as male, female or nonbinary."\footnote{\textit{Saguy, A. (2020)}: Why We Should All Use They/Them Pronouns. \cite{pronouns}} The aim is to support an inclusive approach in science through gender-neutral language. 

In this paper, I'll attempt to follow this suggestion and invite all my readers to do the same for future articles. Thank you!

If you're not sure about the definitions of gender and sex and how to use them, have a look at the definitions\footnote{See \textit{APA (2021)}: Definitions Related to Sexual Orientation. \cite{apaDefinitions}} by the American Psychological Association.

\subsection{Climate Emergency}

As Professor Rahmstorf puts it: "Without immediate, decisive climate protection measures, my children currently attending high school could already experience a 3-degree warmer Earth. No one can say exactly what this world would look like—it would be too far outside the entire experience of human history. But almost certainly, this earth would be full of horrors for the people who would have to experience it."\footnote{\textit{Rahmstorf, A. (2024)}: Climate and Weather at 3 Degrees More. \cite{3dgreesMore}}
