%
%	Einfuehrung
%

\pagebreak
\section{Introduction}

\onehalfspacing

\subsection{Cyber Defense}

Cyber defense protects computer systems and networks from theft or damage to the hardware, software, or electronic data and disruption or misdirection of their services. It is a subset of cybersecurity\footnote{See \textit{Gemini (2025)}: What is Cyber Security. \cite{bardCybersec}} that focuses specifically on defending against cyberattacks.

NIS2, which stands for Network and Information Systems Directive II, is the EU's legislative act to strengthen cybersecurity across the European Union. It sets strict requirements for various sectors to improve security for essential entities. These entities include organizations in critical sectors like energy, transport, waste management, healthcare, and digital infrastructure providers.\footnote{See \textit{NIS 2 Compliant.org (2024)}: Comprehensive Guide to the NIS 2 Directive. \cite{nisGuide}}

One important attack vector that cyber defense has to consider in this framework is the network traffic to and from an application and between different components, which increases significantly when using a micro-service architecture.

\subsection{Micro-Service Applications}

Microservice architecture involves building software applications structured as a collection of small, autonomous services modeled around a business domain.

Compared to a traditional monolithic architecture, function calls within the application can be replaced with API calls across the network. On a Kubernetes platform, this would be network traffic within the cluster.

\subsection{TLS and mTLS}

Transport Layer Security (TLS) is a cryptographic protocol designed to provide secure communication over a network. It is the successor to SSL and is widely used to secure web traffic, email, and other Internet protocols.

mTLS enhances TLS by adding mutual authentication. This means the client and the server must present digital certificates and verify each other's identities before establishing a secure connection.

\subsection{Kubernetes}

Kubernetes, or K8s, is an open-source system designed to automate deploying, scaling, and managing applications built using containers. Containers package software in a standardized unit that includes all the dependencies it needs to run, such as code, libraries, and settings. This makes them portable and efficient.

Kubernetes helps manage these containers by grouping them logically. This makes it easier to track and manage complex applications with many containers. The original inspiration for Kubernetes came from Google's internal container orchestration system, Borg.\footnote{See \textit{Gemini (2025)}: What is Kubernetes. \cite{bardKubernetes}} 

Kubernetes reached the 1.0 milestone in 2015 and was donated to the CNCF in 2016. Its current release is 1.32, we will be using 1.31 in our experiment, but 1.30 was a very special release:

"For the people who built it, for the people who release it, and for the furries who keep all of our clusters online, we present to you Kubernetes v1.30: Uwubernetes, the cutest release to date."\footnote{\textit{Dsouza, A. (2024)}: Kubernetes 1.30. \cite{uwubernetes}}

\begin{figure}[H]
\centering
\caption {Kubernetes 1.30 Release Logo}
\includegraphics[width=0.3\linewidth]{images/k8s-1.30.png}
\label{fig:uwubernetes}
\end{figure}

\subsection{Research Question \& Method}

This paper will examine whether a service mesh can enhance the encryption of intra-application traffic in a sample micro-service application using mTLS.

To do this, we will perform an Experiment with a Kubernetes cluster and cross-check the findings of two Service-Mesh installations.\footnote{See \textit{Genau, L. (2020)}: Ein Experiment in Deiner Abschlussarbeit Durchführen. \cite{expScribbr}}

The goal is to establish whether \href{https://linkerd.io/}{Linkerd} or \href{https://istio.io/}{Istio} can add mTLS to an existing application and determine which uses fewer resources.

\subsection{Gender-neutral Pronouns}

Our society is becoming more open, inclusive, and gender-fluid, and now I think it's time to think about using gender-neutral pronouns in scientific texts, too. Two well-known researchers, Abigail C. Saguy and Juliet A. Williams, both from UCLA, propose to use the singular they/them instead: "The universal singular they is inclusive of people who identify as male, female or nonbinary."\footnote{\textit{Saguy, A. (2020)}: Why We Should All Use They/Them Pronouns. \cite{pronouns}} The aim is to support an inclusive approach in science through gender-neutral language. 

In this paper, I'll attempt to follow this suggestion and invite all my readers to do the same for future articles. Thank you!

If you're not sure about the definitions of gender and sex and how to use them, have a look at the definitions\footnote{See \textit{APA (2021)}: Definitions Related to Sexual Orientation. \cite{apaDefinitions}} by the American Psychological Association.

\subsection{Climate Emergency}

As Professor Rahmstorf puts it: "Without immediate, decisive climate protection measures, my children currently attending high school could already experience a 3-degree warmer Earth. No one can say exactly what this world would look like—it would be too far outside the entire experience of human history. But almost certainly, this earth would be full of horrors for the people who would have to experience it."\footnote{\textit{Rahmstorf, A. (2024)}: Climate and Weather at 3 Degrees More. \cite{3dgreesMore}}
